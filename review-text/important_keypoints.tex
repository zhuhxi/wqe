\documentclass[12pt,a4paper]{ctexart}
\usepackage[a4paper,margin=2.2cm]{geometry}
\usepackage{enumitem}
\usepackage{hyperref}
\setlist[itemize]{leftmargin=1.6em}
\setlist[enumerate]{leftmargin=1.8em}
\hypersetup{colorlinks=true,linkcolor=black,urlcolor=blue}

\title{CNS Review-Text 重要考点(高频版)}
\author{}
\date{\today}

\begin{document}
\maketitle

\section{对比题(必背)}

\begin{enumerate}
  \item \texttt{Bell-LaPadula (BLP)} vs \texttt{Biba}
  \begin{itemize}
    \item \texttt{BLP} 目标是保密性(Confidentiality):\texttt{No Read Up},\texttt{No Write Down}。
    \item \texttt{Biba} 目标是完整性(Integrity):\texttt{No Read Down},\texttt{No Write Up},并且执行也遵循不向上污染(可理解为 \texttt{No Execute Up})。
    \item 结论:两者方向相反,一个防泄密,一个防低可信数据污染高可信数据。
  \end{itemize}

  \item \texttt{PAP} vs \texttt{CHAP}
  \begin{itemize}
    \item \texttt{PAP}:口令(或口令哈希)直接发给服务器,容易被嗅探与重放,安全性弱。
    \item \texttt{CHAP}:挑战-应答,不传输明文口令,每次挑战不同,抗重放更好。
  \end{itemize}

  \item \texttt{AH} vs \texttt{ESP}(IPSec)
  \begin{itemize}
    \item \texttt{AH}:有完整性、认证、防重放;不提供加密;可保护 IP 头完整性。
    \item \texttt{ESP}:提供加密(机密性)并可提供完整性/认证与防重放;不保护外层 IP 头完整性。
    \item 结论:\texttt{ESP} 更常用;也可 \texttt{AH+ESP} 组合。
  \end{itemize}

  \item \texttt{HIDS} vs \texttt{NIDS}
  \begin{itemize}
    \item \texttt{HIDS}:部署在主机上,能看日志/系统调用/文件完整性,适合重点服务器深度检测。
    \item \texttt{NIDS}:部署在网络链路,按流量检测,覆盖面大;但对加密流量可见性差,交换网络下需特殊配置。
  \end{itemize}

  \item \texttt{SSL/TLS} vs \texttt{IPSec VPN}
  \begin{itemize}
    \item \texttt{SSL/TLS}:主要保护传输层之上的应用数据(常见于 HTTPS),更偏应用接入。
    \item \texttt{IPSec VPN}:网络层隧道封装,可保护整个原始 IP 包(ESP 隧道模式),适合站点到站点/全流量保护。
  \end{itemize}

  \item \texttt{Signature-based IDS} vs \texttt{Anomaly-based IDS}
  \begin{itemize}
    \item 特征型:已知攻击检测好、上线快;但对未知攻击弱、需持续维护签名库。
    \item 异常型:能发现新型攻击;但误报率通常更高,正常基线难定义。
  \end{itemize}
\end{enumerate}

\section{流程题(必背顺序)}

\begin{enumerate}
  \item \texttt{CHAP} 流程
  \begin{itemize}
    \item 客户端发 \texttt{userID}。
    \item 服务器发随机 \texttt{challenge}。
    \item 客户端用 \texttt{challenge + password} 计算响应并回传。
    \item 服务器用本地口令做同样计算,比对一致则通过。
  \end{itemize}

  \item \texttt{SSL/TLS} 握手(服务器认证场景)
  \begin{itemize}
    \item 客户端发起连接并请求页面。
    \item 服务器发送证书。
    \item 客户端用本地 CA 公钥链验证证书。
    \item 验证通过后生成会话对称密钥,用服务器公钥加密发给服务器。
    \item 服务器用私钥解密得到会话密钥,后续双方用对称密钥加密通信。
  \end{itemize}

  \item \texttt{Kerberos} 流程(AS/TGS/Service)
  \begin{itemize}
    \item \texttt{C -> AS}:请求票据,拿到 \texttt{TGT} 和 \texttt{C-TGS} 短期密钥材料。
    \item \texttt{C -> TGS}:带 \texttt{TGT} 申请目标服务票据,拿到 \texttt{service ticket} 和 \texttt{C-S} 会话密钥材料。
    \item \texttt{C -> S}:提交服务票据和认证器,服务端验证后建立会话(可双向确认)。
    \item 核心思想:长期密钥尽量短时使用,后续靠票据与短期会话密钥。
  \end{itemize}

  \item \texttt{IKE} 两阶段(IPSec)
  \begin{itemize}
    \item \texttt{Phase 1}:建立受保护的 IKE 信道,完成对等体认证、算法协商、共享密钥交换(主模式/野蛮模式)。
    \item \texttt{Phase 2}:在 Phase 1 保护下协商具体 \texttt{IPSec SA}(AH/ESP 参数、流量选择器、生命周期),用于真实业务流量。
  \end{itemize}

  \item 认证协议新鲜性(Freshness)机制
  \begin{itemize}
    \item 不能只靠加密解决重放问题,必须加 \texttt{Time-stamp}、\texttt{Sequence Number}、\texttt{Nonce} 之一。
    \item 时间戳方案消息轮次少,但依赖时钟同步;\texttt{Nonce} 更通用但需保证随机且不可预测。
  \end{itemize}
\end{enumerate}

\section{设计题(高分模板)}

\begin{enumerate}
  \item 防火墙规则设计模板
  \begin{itemize}
    \item 原则:\texttt{默认拒绝(deny all) + 最小权限 + 显式放行}。
    \item 先写具体规则再写兜底 \texttt{drop any any}(规则匹配通常是自上而下第一条命中)。
    \item 同时考虑源地址、目的地址、端口/服务、状态(stateful)、日志审计。
  \end{itemize}

  \item \texttt{DMZ} 与双防火墙架构
  \begin{itemize}
    \item 外防火墙:只允许公网访问 DMZ 必要服务(如 Web/邮件),其余拒绝。
    \item 内防火墙:严格限制 DMZ 到内网,仅放行明确业务流(如 DNS/邮件中继等)。
    \item 目标:公网可访问服务与内网核心资产物理/逻辑隔离。
  \end{itemize}

  \item 网络加固清单(可直接作答)
  \begin{itemize}
    \item 设备补丁及时更新(交换机/路由器/防火墙)。
    \item 交换机启用端口安全(MAC 绑定、限制非法接入)。
    \item 路由器 ACL 前置过滤恶意或无关流量,减轻边界防火墙压力。
    \item 使用 \texttt{SNMPv3}(认证+加密+访问控制),避免弱管理协议暴露。
    \item 关键链路和安全设备做冗余,避免单点故障。
    \item 主机加固:关闭不必要服务、最小化开放端口、强化认证与审计。
    \item 出站过滤(egress filtering),防止内网被利用对外攻击。
    \item 远程接入走 VPN 并强化端点安全,防止“被感染终端穿透内网”。
  \end{itemize}

  \item IDS/IPS 部署与响应
  \begin{itemize}
    \item 典型放置点:防火墙前后、敏感子网前、关键主机本地。
    \item 响应策略:被动(日志+告警)与主动(断连/重配置/阻断)结合。
    \item NIDS 若要“防御化”(IPS 能力)需串联在线,但要评估性能与单点风险。
  \end{itemize}
\end{enumerate}

\section{常见易错点(考试容易丢分)}
\begin{enumerate}
  \item 把 \texttt{BLP} 和 \texttt{Biba} 读写方向记反。
  \item 误以为 \texttt{CHAP} 会传口令本身。
  \item 误以为 \texttt{AH} 提供加密,或误以为 \texttt{ESP} 保护外层 IP 头完整性。
  \item 把 \texttt{IKE Phase 1} 与 \texttt{Phase 2} 职责混淆(先建安全信道,再建业务 SA)。
  \item 只写“上防火墙”不写规则顺序、默认拒绝、日志与 DMZ 隔离边界。
  \item 忽略加密流量下 \texttt{NIDS} 可见性问题与交换网络镜像/旁路部署条件。
\end{enumerate}

\end{document}
