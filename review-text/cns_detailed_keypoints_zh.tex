\documentclass[12pt,a4paper]{ctexart}
\usepackage[a4paper,margin=2.1cm]{geometry}
\usepackage{hyperref}
\usepackage{enumitem}
\usepackage{array}
\usepackage{booktabs}
\usepackage{longtable}
\usepackage{titlesec}
\setlist[itemize]{leftmargin=1.8em,itemsep=0.35em}
\setlist[enumerate]{leftmargin=2.0em,itemsep=0.35em}
\hypersetup{colorlinks=true,linkcolor=black,urlcolor=blue}
\titleformat{\section}{\large\bfseries}{\thesection}{0.6em}{}
\titleformat{\subsection}{\normalsize\bfseries}{\thesubsection}{0.6em}{}

\title{CNS 详细知识点中文总结(Week 1--14)}
\author{}
\date{\today}

\begin{document}
\maketitle
\tableofcontents
\newpage

\section{使用说明与复习策略}

\subsection{资料范围}
本文基于你目录中的全部课件文本(Week 1, 2, 3--4, 5--6, 7, 8--9, 10--11, 12, 13--14)整理,覆盖课程主线与高频考点。

\subsection{建议复习顺序}
\begin{enumerate}
  \item 先建立主框架:安全目标、机制、威胁与分层。
  \item 再攻克认证与访问控制:\texttt{PAP/CHAP}、\texttt{BLP/Biba}、\texttt{RBAC}。
  \item 再学网络协议与架构:\texttt{Kerberos}、\texttt{IPSec}、\texttt{SSL/TLS}、防火墙与 \texttt{DMZ}。
  \item 最后做综合设计题:规则设计、分区部署、加固与检测响应联动。
\end{enumerate}

\section{课程总框架(Week 1--2 主干)}

\subsection{安全服务与目标}
课程主线围绕以下安全目标展开:
\begin{itemize}
  \item \textbf{Confidentiality(机密性)}:防止未授权读取。
  \item \textbf{Integrity(完整性)}:防止未授权篡改。
  \item \textbf{Availability(可用性)}:系统按需可访问。
  \item \textbf{Authentication(认证)}:证明实体身份。
  \item \textbf{Access Control(访问控制)}:限制谁能访问什么。
  \item \textbf{Non-repudiation(不可否认性)}:行为不可抵赖。
\end{itemize}

\subsection{服务与机制映射(必背)}
\begin{itemize}
  \item 机密性 $\rightarrow$ 加密(对称/非对称)。
  \item 实体认证 $\rightarrow$ 认证协议(挑战应答、证书、票据机制)。
  \item 完整性与数据来源认证 $\rightarrow$ \texttt{MAC} 或数字签名。
  \item 不可否认性 $\rightarrow$ 数字签名(通常需证书体系支撑)。
\end{itemize}

\subsection{安全工程三分法:AAA + PDR}
\begin{itemize}
  \item \texttt{AAA}:\texttt{Authentication, Authorization, Auditing}。
  \item \texttt{PDR}:\texttt{Prevention, Detection, Response}。
  \item 典型认识:预防最理想但成本高;检测更易落地;响应决定恢复效率与损失控制。
\end{itemize}

\subsection{分层思维}
\begin{itemize}
  \item 网络架构分层(OSI/TCP-IP)意味着:\textbf{下层漏洞会被上层继承}。
  \item 安全设计要按层识别攻击面,不要只靠单点工具。
\end{itemize}

\section{密码学与认证基础(Week 1, 3--4)}

\subsection{密码学原语体系}
\begin{itemize}
  \item 对称加密:效率高,密钥分发难。
  \item 非对称加密:分发友好,计算开销大。
  \item 哈希函数:单向摘要,常用于完整性校验与口令存储。
  \item \texttt{MAC}:用共享密钥做消息认证,提供完整性与来源认证。
  \item 数字签名:提供完整性、来源认证与不可否认性。
  \item 证书与 \texttt{PKI}:把“身份”与“公钥”绑定,建立可验证信任链。
\end{itemize}

\subsection{身份认证方法全景}
\begin{itemize}
  \item 知识因子:口令、PIN。
  \item 持有因子:令牌、智能卡。
  \item 生物因子:指纹、人脸、虹膜、行为特征。
  \item 实务建议:敏感系统优先多因子,不应只依赖口令。
\end{itemize}

\subsection{口令安全与常见攻击}
\begin{itemize}
  \item 风险来源:弱口令、重用口令、可预测规则、错误提示泄露、口令传输被嗅探。
  \item 攻击方式:字典/暴力破解、离线哈希碰撞、账号枚举(account harvesting)。
  \item 防护思路:强策略、限速与锁定、统一错误信息、口令哈希加盐、审计告警。
\end{itemize}

\subsection{\texttt{PAP} 与 \texttt{CHAP}(高频对比)}
\begin{itemize}
  \item \texttt{PAP}:口令或口令哈希可被重放,整体偏弱。
  \item \texttt{CHAP}:挑战应答机制,挑战值每次变化,抗重放显著更好。
  \item \texttt{CHAP}流程:\texttt{ID} $\rightarrow$ \texttt{Challenge} $\rightarrow$ \texttt{Response} $\rightarrow$ 验证通过/拒绝。
\end{itemize}

\subsection{SSO 与一次性口令}
\begin{itemize}
  \item \texttt{SSO}优点是体验好;核心风险是单点账号被盗后横向影响大。
  \item \texttt{OTP}(如 RSA SecurID、S/Key)可抵抗窃听后的重放,但通常不能单独解决会话劫持等问题。
\end{itemize}

\subsection{证书认证与私钥保护}
\begin{itemize}
  \item 证书认证强度很大程度取决于\textbf{私钥保护}。
  \item 智能卡/硬件令牌价值:把私钥从通用主机环境中隔离出来。
\end{itemize}

\subsection{生物特征认证(理解型考点)}
\begin{itemize}
  \item 优点:不可遗忘、使用便捷、可提升认证强度。
  \item 难点:误识率权衡、隐私风险、模板泄露不可撤销、伪造攻击(fake biometrics)。
  \item 结论:生物特征适合做增强因子,不宜被神化为“万能认证”。
\end{itemize}

\section{访问控制模型(Week 5--6)}

\subsection{访问控制结构}
\begin{itemize}
  \item 访问控制矩阵:行可看作主体能力(Capabilities),列可看作对象 \texttt{ACL}。
  \item \texttt{ACL}管理的痛点:规模变大后维护复杂,容易引入误配。
  \item 组与角色可降低权限管理复杂度。
\end{itemize}

\subsection{\texttt{DAC/MAC/RBAC}}
\begin{itemize}
  \item \texttt{DAC}:资源拥有者可自主授权,灵活但容易扩散。
  \item \texttt{MAC}:按安全级别强制控制,规则严格。
  \item \texttt{RBAC}:按岗位角色分配权限,工程可维护性通常最好。
\end{itemize}

\subsection{Bell-LaPadula(保密性模型)}
\begin{itemize}
  \item 目标:防止敏感信息向低级别泄露。
  \item 核心规则:\texttt{No Read Up},\texttt{No Write Down}。
  \item 常见考试点:读写方向不要记反。
\end{itemize}

\subsection{Biba(完整性模型)}
\begin{itemize}
  \item 目标:防止低完整性数据污染高完整性对象。
  \item 核心规则:\texttt{No Read Down},\texttt{No Write Up}(执行也避免“向上污染”)。
  \item 与 \texttt{BLP}关系:方向几乎完全相反。
\end{itemize}

\subsection{部分序与分类分区(categories)}
\begin{itemize}
  \item 安全级别不仅有等级(如 Secret/Top Secret),还可叠加分类分区。
  \item 通过“级别 + 分区集合”做支配关系比较,用于更细粒度策略表达。
\end{itemize}

\subsection{Chinese Wall 模型}
\begin{itemize}
  \item 面向利益冲突场景(咨询、金融、审计)。
  \item 关注点是“防跨冲突域信息流动”,强调动态冲突约束。
\end{itemize}

\section{恶意代码体系(Week 7)}

\subsection{主要类型与特征}
\begin{itemize}
  \item Virus:依附宿主传播。
  \item Worm:利用网络漏洞自传播。
  \item Trojan:伪装诱导执行,通常不自传播。
  \item Spyware/Keylogger:窃取行为与敏感输入。
  \item Rootkit:隐藏攻击痕迹与恶意能力。
  \item Backdoor/Botnet:持久控制与批量远程操控。
  \item Logic Bomb:条件触发恶意逻辑。
\end{itemize}

\subsection{防护与检测思路}
\begin{itemize}
  \item 特征库查杀:对已知样本有效,对变种和未知样本有限。
  \item 行为检测:可提升未知威胁发现率,但误报控制难。
  \item 工程策略:最小权限、补丁、端点防护、流量监控、备份恢复、应急预案。
\end{itemize}

\section{认证协议理论与实例(Week 8--11)}

\subsection{实体认证核心概念}
\begin{itemize}
  \item 单向认证:只证明一方身份。
  \item 双向认证:双方互证身份。
  \item 认证是时点行为,若要持续安全,需要后续会话密钥保护。
\end{itemize}

\subsection{认证协议通用要求(高频)}
课件模型强调至少关注:
\begin{enumerate}
  \item 消息来源真实性(确由对方发出)。
  \item 消息新鲜性(非旧报文重放)。
  \item 消息目标绑定(确实发给本方)。
  \item 消息时序绑定(响应与请求强关联,不可被“调包”)。
\end{enumerate}

\subsection{新鲜性机制:时间戳 vs Nonce}
\begin{itemize}
  \item 时间戳优点:消息轮次少,适合客户端-服务端模式。
  \item 时间戳缺点:依赖时钟同步,且需要处理消息关联问题。
  \item Nonce 优点:不依赖全局时钟,抗重放清晰。
  \item Nonce 要求:随机且不可预测,否则协议安全性下降。
\end{itemize}

\subsection{消息关联(Linking Messages)}
\begin{itemize}
  \item 防止攻击者在并发请求中“错配响应”。
  \item 常见做法:引入事务标识(transaction ID)或挑战值绑定。
\end{itemize}

\subsection{Kerberos(重点流程)}
\begin{itemize}
  \item 核心角色:\texttt{AS}、\texttt{TGS}、客户端 \texttt{C}、服务端 \texttt{S}。
  \item 核心思想:用票据与短期会话密钥减少长期密钥暴露。
  \item 典型流程:
  \begin{enumerate}
    \item \texttt{C -> AS}:获取 \texttt{TGT} 与相关密钥材料。
    \item \texttt{C -> TGS}:凭 \texttt{TGT} 申请某服务票据。
    \item \texttt{C -> S}:携票据访问服务并完成会话建立。
  \end{enumerate}
\end{itemize}

\subsection{证书与密钥分发}
\begin{itemize}
  \item 证书是“身份 + 公钥 + 有效期 + 签名”绑定体。
  \item 不同信任域之间依赖交叉证书与证书链验证。
  \item 认证协议常用于会话密钥分发或协商,是后续机密通信基础。
\end{itemize}

\section{网络设备加固与安全架构(Week 12)}

\subsection{设备与管理面加固}
\begin{itemize}
  \item 及时打补丁。
  \item 交换机端口安全(MAC 绑定/限制)。
  \item 路由 \texttt{ACL} 先行过滤,降低边界防火墙压力。
  \item 管理协议优先 \texttt{SNMPv3}(认证、隐私、访问控制)。
\end{itemize}

\subsection{安全网络设计原则}
\begin{itemize}
  \item 先有安全策略,再落地架构与控制。
  \item 在设计阶段内建安全,而不是后贴补丁。
  \item 平衡三角:性能、可用性、安全。
  \item 减少单点故障:防火墙/路由/链路冗余。
\end{itemize}

\subsection{边界与内部并重}
\begin{itemize}
  \item 仅靠外部边界防护不够,要防内部发起攻击。
  \item 关键实践:\texttt{DMZ} 分区、主机加固、出站过滤、远程接入端点治理。
  \item VPN 终端受感染会带来“加密隧道穿透边界”的风险。
\end{itemize}

\section{边界防护、VPN 与检测响应(Week 13--14)}

\subsection{防火墙基础与策略设计}
\begin{itemize}
  \item 防火墙本质:默认拒绝,按策略放行并审计。
  \item 常见判定维度:源/目的地址、端口服务、连接状态、应用内容。
  \item 常见动作:\texttt{Accept}、\texttt{Drop}、\texttt{Reject}、\texttt{Authenticate}。
\end{itemize}

\subsection{防火墙类型}
\begin{itemize}
  \item 包过滤(网络/传输层):
  \begin{itemize}
    \item 优点:高性能、部署广。
    \item 局限:深层应用语义可见性弱。
  \end{itemize}
  \item 应用层代理防火墙:
  \begin{itemize}
    \item 优点:协议理解更深,策略更细。
    \item 局限:性能开销更高。
  \end{itemize}
  \item 混合型:工程上最常见,按业务组合能力。
\end{itemize}

\subsection{架构与规则(设计题高频)}
\begin{itemize}
  \item 单防火墙架构:部署简单,隔离能力有限。
  \item 双防火墙架构:外层守公网,内层守内网,\texttt{DMZ} 位于中间,隔离更强。
  \item 规则顺序遵循首条匹配(first match):\textbf{具体规则在上,泛化规则在下,最后兜底拒绝}。
\end{itemize}

\subsection{IPSec 总结}
\begin{itemize}
  \item 协议族:\texttt{AH}、\texttt{ESP}、\texttt{IKE}。
  \item 运行层:网络层,适合站点到站点与全流量保护。
  \item 模式:
  \begin{itemize}
    \item 传输模式:保护上层负载,外层 IP 头保留。
    \item 隧道模式:封装整个原始 IP 包,更适合 VPN。
  \end{itemize}
\end{itemize}

\subsection{\texttt{AH} vs \texttt{ESP}}
\begin{itemize}
  \item \texttt{AH}:认证、完整性、防重放;不加密;可保护 IP 头完整性。
  \item \texttt{ESP}:加密为主,也可提供完整性与认证;常见于 VPN。
  \item 两者可组合使用,按场景取舍。
\end{itemize}

\subsection{安全关联与 IKE 两阶段}
\begin{itemize}
  \item \texttt{SA}(Security Association)是 IPSec 安全上下文核心。
  \item \texttt{Phase 1}:建立受保护信道,完成对等体认证与密钥交换(Main/Aggressive)。
  \item \texttt{Phase 2}:协商具体业务流 \texttt{IPSec SA} 参数并周期更新。
\end{itemize}

\subsection{SSL/TLS 主线}
\begin{itemize}
  \item 位置:位于应用与 TCP 之间,保护应用层数据通道。
  \item 服务:服务器认证、加密、消息完整性(客户端认证可选)。
  \item 阶段:握手阶段协商算法与密钥,数据阶段进行加密传输。
\end{itemize}

\subsection{\texttt{SSL/TLS} vs \texttt{IPSec VPN}}
\begin{itemize}
  \item \texttt{SSL/TLS}:更偏应用接入,按会话/应用保护。
  \item \texttt{IPSec VPN}:更偏网络层隧道,按链路/网段保护。
  \item 选型应基于业务范围、终端可控性、运维复杂度和合规要求。
\end{itemize}

\subsection{IDS/IPS(检测与防御联动)}
\begin{itemize}
  \item \texttt{HIDS}:主机视角,适合关键资产深度检测。
  \item \texttt{NIDS}:网络视角,覆盖广,但对加密流量可见性受限。
  \item 检测模型:
  \begin{itemize}
    \item 特征检测:已知威胁效果好。
    \item 异常检测:可发现未知威胁,但误报率通常更高。
  \end{itemize}
  \item 响应方式:被动(日志告警)与主动(断连、重配置、阻断)。
  \item NIDS 变 IPS(串联在线)要评估性能、可用性与误阻断风险。
\end{itemize}

\section{综合题答题模板(可直接套用)}

\subsection{模板 A:防火墙/DMZ 设计题}
\begin{enumerate}
  \item 先说明安全域:Internet、DMZ、Intranet、管理区。
  \item 写原则:默认拒绝、最小权限、纵深防御、审计留痕。
  \item 列核心放行:公网到 DMZ 的必要服务、内网到外网的业务流、DMZ 到内网最小化通道。
  \item 列拒绝策略:其余全拒绝,出站过滤,伪造源地址过滤。
  \item 写高可用:双机热备、双链路、配置一致性与变更审计。
\end{enumerate}

\subsection{模板 B:认证协议分析题}
\begin{enumerate}
  \item 指出协议目标:单向/双向认证,是否附带会话密钥建立。
  \item 检查四点:来源真实性、新鲜性、目标绑定、消息关联。
  \item 标注攻击面:重放、反射、并发错配、中间人、时钟偏差。
  \item 给修复建议:引入 Nonce/时间戳、事务 ID、双向绑定、密钥更新周期。
\end{enumerate}

\subsection{模板 C:IDS/IPS 部署题}
\begin{enumerate}
  \item 明确资产分级:互联网出口、边界防火墙后、敏感子网前、关键主机本地。
  \item 按位置说明传感器:NIDS 与 HIDS 组合部署。
  \item 给响应分层:先被动告警,关键场景再主动阻断。
  \item 提醒代价:在线阻断会带来性能、误报与单点问题。
\end{enumerate}

\section{高频必背清单(考前 15 分钟)}

\begin{enumerate}
  \item \texttt{BLP}:\texttt{No Read Up, No Write Down};\texttt{Biba}:\texttt{No Read Down, No Write Up}。
  \item \texttt{PAP} 弱,\texttt{CHAP} 挑战应答抗重放。
  \item 认证协议不只要“能验身份”,还要“防重放、能关联请求响应”。
  \item 新鲜性三件套:时间戳、逻辑序号、Nonce。
  \item Kerberos:\texttt{AS -> TGS -> Service} 票据链路。
  \item \texttt{AH} 不加密、\texttt{ESP} 可加密;隧道模式保护更完整。
  \item \texttt{IKE Phase 1} 建安全信道,\texttt{Phase 2} 建业务 SA。
  \item \texttt{SSL/TLS} 偏应用层保护,\texttt{IPSec VPN} 偏网络层保护。
  \item 防火墙规则:首条匹配,先具体后泛化,最后默认拒绝。
  \item \texttt{HIDS + NIDS} 组合优于单点;异常检测强在未知威胁,弱在误报控制。
\end{enumerate}

\section{常见失分点}
\begin{itemize}
  \item 把 \texttt{BLP/Biba} 的读写方向写反。
  \item 说 \texttt{CHAP} 会传输口令本身。
  \item 误写 \texttt{AH} 提供加密,或误写 \texttt{ESP} 保护外层 IP 头完整性。
  \item 把 \texttt{IKE} 两阶段职责混淆。
  \item 防火墙题只写“部署防火墙”不写规则顺序、默认拒绝、日志和分区边界。
  \item 忽略加密流量环境下 \texttt{NIDS} 的可见性限制。
\end{itemize}

\section*{结语}
这份总结可直接用于三类题型:
\begin{itemize}
  \item 对比题:模型、协议、检测方法的差异与边界。
  \item 流程题:认证握手、票据交换、IKE 两阶段。
  \item 设计题:规则、分区、加固、检测与响应的整体方案。
\end{itemize}

\end{document}
