\documentclass[12pt,a4paper]{ctexart}
\usepackage[a4paper,margin=2.1cm]{geometry}
\usepackage{enumitem}
\usepackage{hyperref}
\usepackage{titlesec}
\setlist[itemize]{leftmargin=1.8em,itemsep=0.35em}
\setlist[enumerate]{leftmargin=2.0em,itemsep=0.35em}
\titleformat{\section}{\large\bfseries}{\thesection}{0.6em}{}
\titleformat{\subsection}{\normalsize\bfseries}{\thesubsection}{0.6em}{}
\hypersetup{colorlinks=true,linkcolor=black,urlcolor=blue}

\title{网络安全每周白话学习手册(零基础版)}
\author{}
\date{\today}

\begin{document}
\maketitle
\tableofcontents
\newpage

\section{怎么用这份手册}
\begin{itemize}
  \item 这份文档按 Week 1 到 Week 14 排好,适合没学过网络安全的人。
  \item 每周固定三部分:\textbf{本周要学会什么}、\textbf{术语白话+例子}、\textbf{本周学习任务}。
  \item 你不需要一次看完。建议 1 周看 1 章,先理解,再记术语。
\end{itemize}

\section{Week 1:先知道“安全到底保护什么”}
\subsection{本周要学会什么}
\begin{itemize}
  \item 网络安全不只是“防黑客”,而是保护信息系统能可靠运行。
  \item 核心目标有 6 个:机密性、完整性、可用性、认证、授权、不可否认。
\end{itemize}

\subsection{术语白话 + 例子}
\begin{enumerate}
  \item \textbf{Confidentiality(机密性)}:不该看的人看不到。例子:工资表只给 HR 看。
  \item \textbf{Integrity(完整性)}:内容不能被偷偷改。例子:转账 1000 不能被改成 10000。
  \item \textbf{Availability(可用性)}:该用的时候能用。例子:挂号系统不能总宕机。
  \item \textbf{Authentication(认证)}:证明“你是谁”。例子:密码 + 手机验证码登录。
  \item \textbf{Authorization(授权)}:登录后“你能做什么”。例子:你能看报表但不能删数据。
  \item \textbf{Non-repudiation(不可否认)}:做过就不能赖。例子:数字签名后的合同不能否认签过。
\end{enumerate}

\subsection{本周学习任务}
\begin{itemize}
  \item 用自己的话解释这 6 个目标(每个一句话)。
  \item 试着给每个目标再举 1 个生活中的例子。
\end{itemize}

\section{Week 2:先学“怎么防”,再学“防哪里”}
\subsection{本周要学会什么}
\begin{itemize}
  \item 安全工作有三步:预防、检测、响应(PDR)。
  \item 管理上有 AAA:认证、授权、审计。
  \item 网络是分层的,攻击也会按层出现。
\end{itemize}

\subsection{术语白话 + 例子}
\begin{enumerate}
  \item \textbf{Prevention}:事前阻断。例子:防火墙先把高危端口封掉。
  \item \textbf{Detection}:发现异常。例子:IDS 检测到可疑扫描并告警。
  \item \textbf{Response}:出事后处置。例子:中毒主机隔离、重装、恢复备份。
  \item \textbf{AAA}:认证身份、授予权限、记录日志。例子:VPN 登录全流程。
  \item \textbf{OSI/TCP-IP 分层}:不同层有不同风险。例子:DNS 欺骗属于应用层风险。
\end{enumerate}

\subsection{本周学习任务}
\begin{itemize}
  \item 把你常用一个系统(如网盘)用 PDR 思路拆解一次。
  \item 想一想它的 AAA 分别在哪里体现。
\end{itemize}

\section{Week 3:账号口令安全(第一部分)}
\subsection{本周要学会什么}
\begin{itemize}
  \item 密码为什么会被破解。
  \item 为什么不能明文存密码,为什么要哈希与加盐。
  \item 如何减少账号枚举(account harvesting)风险。
\end{itemize}

\subsection{术语白话 + 例子}
\begin{enumerate}
  \item \textbf{Password Cracking}:猜密码。例子:先跑弱口令字典,再暴力枚举。
  \item \textbf{Hash}:单向摘要。例子:文件改一个字符,哈希值就大变。
  \item \textbf{Salt}:给密码加随机“调料”再哈希。例子:同样是 123456,数据库里哈希也不同。
  \item \textbf{Account Harvesting}:探测哪些账号存在。例子:登录页提示“用户不存在”泄露信息。
\end{enumerate}

\subsection{本周学习任务}
\begin{itemize}
  \item 写出 5 条强密码规则(长度、复杂度、唯一性等)。
  \item 解释“为什么只哈希不加盐也不够”。
\end{itemize}

\section{Week 4:账号口令安全(第二部分)与高级认证}
\subsection{本周要学会什么}
\begin{itemize}
  \item 认证协议基础:PAP vs CHAP。
  \item SSO、OTP、证书认证、生物认证各自优缺点。
\end{itemize}

\subsection{术语白话 + 例子}
\begin{enumerate}
  \item \textbf{PAP}:认证材料可重放,安全性弱。例子:抓到包后可能直接复用。
  \item \textbf{CHAP}:挑战应答,每次挑战不同。例子:门卫每次问不同口令。
  \item \textbf{SSO}:一次登录多个系统。例子:公司门户一登,OA 和邮箱都可用。
  \item \textbf{OTP}:一次性密码。例子:动态验证码每 60 秒变一次。
  \item \textbf{Certificate Authentication}:用证书证明身份。例子:VPN 只允许企业证书设备接入。
  \item \textbf{Biometrics}:生物特征认证。例子:指纹或人脸解锁。
\end{enumerate}

\subsection{本周学习任务}
\begin{itemize}
  \item 能口述 CHAP 四步流程。
  \item 解释为什么“方便的 SSO”也有“单点失守”风险。
\end{itemize}

\section{Week 5:访问控制基础}
\subsection{本周要学会什么}
\begin{itemize}
  \item 什么是访问控制矩阵。
  \item ACL 与 Capability 的视角差异。
  \item DAC、MAC、RBAC 的区别。
\end{itemize}

\subsection{术语白话 + 例子}
\begin{enumerate}
  \item \textbf{Access Control Matrix}:谁对什么资源有何权限的总表。
  \item \textbf{ACL}:从“资源角度”看谁能访问我。
  \item \textbf{Capability}:从“用户角度”看我能访问谁。
  \item \textbf{DAC}:资源所有者自己授权。例子:你分享自己网盘文件。
  \item \textbf{MAC(访问控制)}:按系统强制规则授权。例子:涉密系统按密级。
  \item \textbf{RBAC}:按角色授权。例子:财务角色默认有报销审批权限。
\end{enumerate}

\subsection{本周学习任务}
\begin{itemize}
  \item 把“学校教务系统”画成一个小型权限矩阵。
  \item 说明为什么企业里 RBAC 常比直接 ACL 更好维护。
\end{itemize}

\section{Week 6:经典安全模型}
\subsection{本周要学会什么}
\begin{itemize}
  \item BLP 与 Biba 的方向差异。
  \item 角色模型与利益冲突模型(Chinese Wall)。
\end{itemize}

\subsection{术语白话 + 例子}
\begin{enumerate}
  \item \textbf{BLP}:防泄密。口诀:\texttt{No Read Up, No Write Down}。
  \item \textbf{Biba}:防污染。口诀:\texttt{No Read Down, No Write Up}。
  \item \textbf{Partial Ordering}:不是所有分类都能直接比较高低。
  \item \textbf{Chinese Wall}:防利益冲突访问。例子:顾问看过 A 机密后不能看 B 竞品机密。
\end{enumerate}

\subsection{本周学习任务}
\begin{itemize}
  \item 用一句话区分 BLP 和 Biba 的“保护目标”。
  \item 每个模型各举 1 个业务场景。
\end{itemize}

\section{Week 7:恶意软件(Malware)}
\subsection{本周要学会什么}
\begin{itemize}
  \item 病毒、蠕虫、木马、间谍软件、Rootkit、后门、僵尸网络的区别。
  \item 检测为什么不能只靠一种手段。
\end{itemize}

\subsection{术语白话 + 例子}
\begin{enumerate}
  \item \textbf{Virus}:依附文件传播。例子:感染文档后继续感染别的文档。
  \item \textbf{Worm}:可自动通过网络扩散。例子:一台中招后扫网段继续感染。
  \item \textbf{Trojan}:伪装软件骗你安装。例子:破解工具里藏后门。
  \item \textbf{Spyware/Keylogger}:偷行为数据和键盘输入。
  \item \textbf{Rootkit}:隐藏恶意痕迹。例子:系统看不到它但它在运行。
  \item \textbf{Botnet}:大量肉鸡被统一控制发起攻击。
\end{enumerate}

\subsection{本周学习任务}
\begin{itemize}
  \item 比较 Virus 和 Worm 的传播方式。
  \item 解释“为什么 Rootkit 更难查”。
\end{itemize}

\section{Week 8:认证协议理论(上)}
\subsection{本周要学会什么}
\begin{itemize}
  \item 认证协议不只是“验证身份”,还要防重放和错配。
  \item 单向认证与双向认证差异。
\end{itemize}

\subsection{术语白话 + 例子}
\begin{enumerate}
  \item \textbf{Entity Authentication}:确认通信实体身份。
  \item \textbf{Unilateral Authentication}:只验证一方。
  \item \textbf{Mutual Authentication}:双方互相验证。
  \item \textbf{Protocol Requirements}:来源真实性、新鲜性、目标绑定、时序绑定。
\end{enumerate}

\subsection{本周学习任务}
\begin{itemize}
  \item 试说出“只做身份验证还不够”的原因。
  \item 举例说明为什么需要“消息目标绑定”。
\end{itemize}

\section{Week 9:认证协议理论(下)}
\subsection{本周要学会什么}
\begin{itemize}
  \item 新鲜性机制:时间戳、序列号、Nonce。
  \item 链接请求与响应(linking messages)的意义。
\end{itemize}

\subsection{术语白话 + 例子}
\begin{enumerate}
  \item \textbf{Replay Attack}:重放旧包骗系统。
  \item \textbf{Time-stamp}:看时间是否过期。
  \item \textbf{Sequence Number}:用递增序号防乱序与重放。
  \item \textbf{Nonce}:一次性随机挑战值。
  \item \textbf{Transaction ID}:把请求和响应一一绑定,防串包。
\end{enumerate}

\subsection{本周学习任务}
\begin{itemize}
  \item 比较时间戳和 Nonce 各自优缺点。
  \item 用一句话解释“加密本身不能自动防重放”。
\end{itemize}

\section{Week 10:安全协议实例}
\subsection{本周要学会什么}
\begin{itemize}
  \item 理解典型认证协议消息来回设计思路。
  \item 明白“协议属性”如何被验证。
\end{itemize}

\subsection{术语白话 + 例子}
\begin{enumerate}
  \item \textbf{Time-stamp Protocol}:轮次少,但依赖时钟同步。
  \item \textbf{Nonce Protocol}:更灵活,但要有可靠随机数。
  \item \textbf{Property Linking}:响应必须对应本次请求。
\end{enumerate}

\subsection{本周学习任务}
\begin{itemize}
  \item 画一个最小挑战应答流程图(3~4步)。
  \item 指出流程里哪一步在防重放。
\end{itemize}

\section{Week 11:Kerberos 与证书链}
\subsection{本周要学会什么}
\begin{itemize}
  \item Kerberos 的角色与三段式流程。
  \item 证书链如何建立跨域信任。
\end{itemize}

\subsection{术语白话 + 例子}
\begin{enumerate}
  \item \textbf{Kerberos}:票据式认证系统。
  \item \textbf{AS}:先验证你身份并给初始票据。
  \item \textbf{TGS}:给你发访问具体服务的票据。
  \item \textbf{TGT}:访问 TGS 的通行证。
  \item \textbf{Session Key}:一次会话临时密钥。
\end{enumerate}

\subsection{本周学习任务}
\begin{itemize}
  \item 口述 Kerberos 三步:\texttt{C->AS, C->TGS, C->S}。
  \item 解释“为什么要用临时会话密钥而不是长期密钥一直用”。
\end{itemize}

\section{Week 12:设备加固与网络架构设计}
\subsection{本周要学会什么}
\begin{itemize}
  \item 交换机/路由器加固、SNMPv3、边界和内部并重。
  \item 安全设计要考虑可用性,不要单点故障。
\end{itemize}

\subsection{术语白话 + 例子}
\begin{enumerate}
  \item \textbf{Network Hardening}:收紧配置、打补丁、减少攻击面。
  \item \textbf{Port Security}:限制交换机端口允许的设备。
  \item \textbf{SNMPv3}:带认证和加密的网络管理协议。
  \item \textbf{DMZ}:公网服务隔离区。
  \item \textbf{Outbound Filtering}:限制内网向外发可疑流量。
\end{enumerate}

\subsection{本周学习任务}
\begin{itemize}
  \item 写一份 8 条网络加固清单(补丁、端口、账户、日志等)。
  \item 画一个“内网- DMZ -互联网”简图。
\end{itemize}

\section{Week 13:防火墙、VPN、IPSec}
\subsection{本周要学会什么}
\begin{itemize}
  \item 防火墙规则如何设计(默认拒绝、最小权限、规则顺序)。
  \item IPSec 关键术语(AH/ESP、模式、SA、IKE)。
\end{itemize}

\subsection{术语白话 + 例子}
\begin{enumerate}
  \item \textbf{Firewall}:按规则决定流量放行/拒绝。
  \item \textbf{First Match}:规则从上到下,命中即停。
  \item \textbf{AH}:认证/完整性/防重放,不加密。
  \item \textbf{ESP}:可加密,VPN 常用。
  \item \textbf{Transport vs Tunnel}:一个保上层负载,一个包整个原始 IP 包。
  \item \textbf{SA/SPI}:安全参数集合及其编号。
  \item \textbf{IKE Phase 1/2}:先建安全协商通道,再建业务 SA。
\end{enumerate}

\subsection{本周学习任务}
\begin{itemize}
  \item 给出一组最小防火墙策略(放 443,其他拒绝)。
  \item 解释 AH 和 ESP 的核心区别。
\end{itemize}

\section{Week 14:SSL/TLS 与 IDS/IPS}
\subsection{本周要学会什么}
\begin{itemize}
  \item SSL/TLS 握手核心步骤与 VPN 的区别。
  \item IDS/IPS 的部署方式和取舍。
\end{itemize}

\subsection{术语白话 + 例子}
\begin{enumerate}
  \item \textbf{TLS Handshake}:协商算法、验身份、建会话密钥。
  \item \textbf{SSL/TLS vs IPSec VPN}:前者偏应用会话,后者偏网络隧道。
  \item \textbf{HIDS}:主机侧检测。
  \item \textbf{NIDS}:网络侧检测。
  \item \textbf{Signature Detection}:抓已知攻击特征。
  \item \textbf{Anomaly Detection}:抓“偏离正常”的行为。
  \item \textbf{IPS}:在检测基础上可主动阻断。
\end{enumerate}

\subsection{本周学习任务}
\begin{itemize}
  \item 说出 HIDS 和 NIDS 的差别与适用场景。
  \item 解释“为什么异常检测会更容易误报”。
\end{itemize}

\section{最后的学习建议(给零基础)}
\begin{itemize}
  \item 先会讲人话版本,再背术语,不要反过来。
  \item 每周做 3 件事:\textbf{看懂概念}、\textbf{会举例子}、\textbf{能画流程图}。
  \item 记忆优先级:\texttt{BLP/Biba}、\texttt{PAP/CHAP}、\texttt{Kerberos}、\texttt{AH/ESP}、\texttt{IKE两阶段}、\texttt{HIDS/NIDS}。
\end{itemize}

\end{document}
